\documentclass[11pt, a4paper]{article}

\usepackage[english]{babel}
\usepackage[utf8]{inputenc}
\usepackage{amsmath}
\usepackage[colorinlistoftodos]{todonotes}
\usepackage{makecell}
\usepackage{multirow}
\usepackage{caption}
\usepackage{subcaption}
\usepackage{graphicx}
\usepackage{hyperref}
\usepackage{float}
\usepackage[all]{hypcap}
\usepackage[space]{grffile}
\usepackage{enumitem}
\usepackage{bm}
\usepackage{bbm}
\usepackage{algorithm}
\usepackage{algorithmic}
\usepackage{nccmath, mathtools}
\usepackage{amsthm,amssymb}
\usepackage{listings}

\newlist{questions}{enumerate}{1}
\setlist[questions, 1]{label = \arabic*}
\newlist{bonus}{enumerate}{1}
\setlist[bonus, 1]{label = Bonus \arabic*}

\definecolor{mygreen}{rgb}{0,0.6,0}
\definecolor{mygray}{rgb}{0.5,0.5,0.5}
\definecolor{mymauve}{rgb}{0.58,0,0.82}

\lstset{ 
	backgroundcolor=\color{white},   % choose the background color; you must add \usepackage{color} or \usepackage{xcolor}; should come as last argument
	basicstyle=\footnotesize,        % the size of the fonts that are used for the code
	breakatwhitespace=false,         % sets if automatic breaks should only happen at whitespace
	breaklines=true,                 % sets automatic line breaking
	captionpos=b,                    % sets the caption-position to bottom
	commentstyle=\color{mygreen},    % comment style
	deletekeywords={...},            % if you want to delete keywords from the given language
	escapeinside={\%*}{*)},          % if you want to add LaTeX within your code
	extendedchars=true,              % lets you use non-ASCII characters; for 8-bits encodings only, does not work with UTF-8
	frame=single,	                   % adds a frame around the code
	keepspaces=true,                 % keeps spaces in text, useful for keeping indentation of code (possibly needs columns=flexible)
	keywordstyle=\color{blue},       % keyword style
	language=Octave,                 % the language of the code
	morekeywords={*,...},            % if you want to add more keywords to the set
	numbers=left,                    % where to put the line-numbers; possible values are (none, left, right)
	numbersep=5pt,                   % how far the line-numbers are from the code
	numberstyle=\tiny\color{mygray}, % the style that is used for the line-numbers
	rulecolor=\color{black},         % if not set, the frame-color may be changed on line-breaks within not-black text (e.g. comments (green here))
	showspaces=false,                % show spaces everywhere adding particular underscores; it overrides 'showstringspaces'
	showstringspaces=false,          % underline spaces within strings only
	showtabs=false,                  % show tabs within strings adding particular underscores
	stepnumber=2,                    % the step between two line-numbers. If it's 1, each line will be numbered
	stringstyle=\color{mymauve},     % string literal style
	tabsize=2,	                   % sets default tabsize to 2 spaces
	title=\lstname                   % show the filename of files included with \lstinputlisting; also try caption instead of title
}

% Adjust margins
\addtolength{\oddsidemargin}{-0.75in}
\addtolength{\evensidemargin}{-0.75in}
\addtolength{\textwidth}{1.5in}
\addtolength{\topmargin}{-.5in}
\addtolength{\textheight}{1.5in}
\setlength\parindent{0pt}
\setlength{\parskip}{5pt}

\title{CS 536 : Final Project - Data Completion and Interpolation}
\author{Haoyang Zhang, Han Wu, Shengjie Li}
\date{\today}

\begin{document}
\maketitle

\section{Introduction, group members and division of workload}
\label{sec:Introduction}

In this group project, we implemented an autocoder for interpolating missing features from features we have and achieved . \\
\begin{tabular}{| p{2.7cm} | p{\textwidth -3.5cm} |}
	\hline
	\makecell[c]{Name \\ NetID} & Workload \\
	\hline
	\makecell[c]{Han Wu \\ hw436} & {Fine-tuned the parameters of our model. Did some experiments for the evaluation of our model. Wrote part of the report.} \\
	\hline
	\makecell[c]{Haoyang Zhang \\ hz333} & {Analyzed and wrote scripts to clean the data. Wrote scripts to restore human-friendly data from the output of our model. Wrote part of the report. } \\
	\hline
	\makecell[c]{Shengjie Li \\ sl1560} & {Implemented the basics of neural networks including back-propagation and several loss functions and activation functions. Wrote part of the report.} \\
	\hline
\end{tabular}

\section{Prerequisites}
\label{sec:Prerequisites}
\begin{enumerate}
	\item {How to represent or process the data. Data features may contain a number of diverse data types (real values, integer values, categorical or binary values, ordered categorical values, open/natural language responses). How can you represent these for processing and prediction?}
	
	\item{How to model the problem of interpolation. What are the inputs, what are the outputs? An important if subtle question to consider here - what does it mean to predictor or interpolate a missing feature?}
	
	\item{Model selection. What kind of model or models do you want to consider?}
	\begin{itemize}
		\item {A transfer learning approach:}
		\item {Autoencoders with RNN:}
	\end{itemize}
	
	\item{Quantifying loss or error. How can you quantify how good a model is, how to measure its loss/error? This is important not only in terms of evaluating your model, but in terms of training as well - how can you refine and improve your model without a way of comparing them?}
	\par{Because we are using an autoencoder, we are measuring the reconstruction errors. For the reason that there are many data types, we are treating the data over different loss functions. For real values and integer values, we are using mean squared error $ L(\theta) = \frac{1}{m}\sum_{i=1}^{m}(y^i - p^i)^2 $ (m denotes the number of data points) and root-mean-squared error}
	
	\item{Training. What kind of training algorithm can you apply to your model(s)? What design choices do you have to make here?}
	
	\item{Feature selection. It is frequently useful in learning problems to focus on specific features and exclude others, to try to eliminate spurious features and focus on what matters. How can that be applied here?}
	
	\item{Validation. How can you prevent or avoid over-fitting? Can you apply the usual training/testing/validation paradigm to this problem? How do you choose the training or testing data? Note that a record won’t need to be complete to still be useful, potentially, in interpolation. Can cross-validation be applied here? This can be especially important when the data set is not overwhelmingly large and data must be used carefully.}
	
	\item{Evaluation. How good is your final model? How can you evaluate this? What are the limits and strengths of your model - how many features does a new record actually need to be able to interpolate well?}
\end{enumerate}


\section{Requirements}
\label{sec:Requirements}


\section{Bonus}
\label{sec:Bonus}

\end{document}